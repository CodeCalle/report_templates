\section*{Related Work}
    The paper by Kelly and Weaver \cite{kelly2004goal} introduces the Goal Structuring Notation (GSN) as a solution to improve the structure, rigor, and clarity of safety arguments in safety cases. It highlights the importance of safety cases in various industries and emphasizes the need for clear communication of safety arguments. GSN is presented as a graphical argumentation notation that represents safety argument elements and their relationships. The paper discusses GSN's application in safety-critical industries and introduces extensions to aid in argument maintenance, construction, reuse, and assessment. The paper also acknowledges a potential weakness of using GSN-based argumentation. These weaknesses could include challenges such as subjectivity in assurance, industry-specific variations, and the generalization of GSN's applicability. However, despite the potential weaknesses, the GSN-based argumentation is a suitable option for my project as it aligns well with the goal of creating a structured safety case. The use of the GSN will provide a clear and concise argumentation structure that can be effectively used for communicating safety arguments.
    
    The paper by Palin et al. \cite{palin2011iso} gives an overview of the current state of using safety assurance cases within the automotive industry. Currently there is no requirement of any explicit argumentation to assure the safety of the system, but instead relies on the compliance with national and international standards and guidelines. However, introducing the standard ISO 26262 where requirements for functional safety cases have been introduced, the interest for a formal method has increased in interest. The paper acknowledges challenges of introducing the requirement for a safety case. One challenge is that in some cases it can be reduntant to create formal argumentation since the system already complies with other standards and no further proof might be necessary. However, a safety case created with methods such as GSN can be used to provide an overview of the safety by providing a structured argumentation, traceability between safety goals, evidence and rationales, and visualization of the safety case which can be effectively communicated to various stakeholders of different expertise. 
    
    After using the GSN-based argumentation in my own project, it is clear that the GSN notation is a suitable option for creating a structured safety case. It should strenghten the the case for using a formal method for safety assurance and provide a clear and concise argumentation structure.

    