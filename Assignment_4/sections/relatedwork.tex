\section*{Related Work}
    The introduction of \ac{gsn} in safety cases aims to enhance argumentation structure, rigor, and clarity. \ac{gsn}, presented as a graphical notation, addresses the need for effective communication of safety arguments in various industries. Despite acknowledging potential weaknesses, \ac{gsn} is considered a suitable option for creating clear and concise structured safety cases. 

    Kelly and Weaver \cite{kelly2004goal} introduces \ac{gsn} as a solution to improve the structure, rigor, and clarity of safety arguments in safety cases. It highlights the importance of safety cases in various industries and emphasizes the need for clear communication of safety arguments. The paper discusses \ac{gsn}'s application in safety-critical industries and introduces extensions to aid in argument maintenance, construction, reuse, and assessment. The paper acknowledges potential weaknesses of using \ac{gsn}-based argumentation. These could be challenges such as subjectivity in assurance, industry-specific variations, and the generalization of \ac{gsn}'s applicability. Palin et al. \cite{palin2011iso} gives an overview of the current state of using safety assurance cases within the automotive industry. Previously there has been no requirement of any explicit argumentation to assure safety of the system, but instead relied on the compliance with various standards and guidelines. However, introducing the standard ISO 26262 where requirements for functional safety cases have been introduced, the interest for a formal argumentation method has increased. The paper acknowledges challenges of introducing the requirement for a safety case. One challenge is that in some cases it can be redundant to create formal argumentation since the system already complies with other standards and no further proof might be necessary. However, by practically applying \ac{gsn} in my own project, it is further proven that the notion of a safety case created with methods such as \ac{gsn} can be used to provide an overview of the safety by creating a structured argumentation, traceability between safety goals and evidence, and visualization of the safety case which can be effectively communicated to various stakeholders of different expertise. The \ac{gsn} seem to be especially useful in the automotive sector, where safety assurance traditionally relies on compliance with standards, \ac{gsn} offers a formal method to present a well-structured safety case, overcoming challenges and providing traceability for effective communication with various stakeholders.